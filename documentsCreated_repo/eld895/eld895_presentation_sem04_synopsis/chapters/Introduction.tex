\section[Introduction]{Introduction}
\label{sec:introduction}


\begin{frame}[fragile]{Introduction}
	\begin{tabularx}{\textwidth}{
			@{\hspace{1.5em}}% Space for left bullet
			>{\leavevmode\raggedright}% Left bullet + formatting of column
			X% Left column specification
			@{\quad\hspace{1.5em}}% Space between columns + right bullet space
			>{\leavevmode\raggedright\arraybackslash}% Right bullet + formatting of column
			X% Right column specification
			@{}% No column space on right
		}
		\textbf{Transient Stability} & \textbf{Steady State Stability}\\
		\toprule
		A sudden, out-of-trend, high magnitude change in a state variable(s) causes blackouts. & 
		Accumulation of several seemingly minor trends in state variables over time, ultimately leading to a \textcolor{red}{critical point} where a small change could cause blackouts.\\
		Causes are generally tangible, including sudden failure of generator or transformer or transmission line or due to corrective protection mechanisms. &
		Causes may not be tangible. Some documented causes include: accumulation of renewable generation `noise', measurement noise, difference in supply demand of powers in a grid after the latest power dispatch.\\
		\bottomrule
	\end{tabularx}
\end{frame}

\begin{frame}{Introduction}
	While modeling every significant possible source of stochastic disturbance can be difficult or perhaps even outright impossible, at least their detection can be made through model free data-driven statistical analysis, enabling early detection of grid stability problems for a timely course-corrective action \cite{schafer01, sanchez01, ghanvati01}.
\end{frame}

\begin{frame}{Introduction}
	Bifurcation Theory \cite{nathanKutzNotesOnBifurcationTheoryAndNormalForms, rosehartBifurcationAnalysisOfVariousPowerSystemModels, chenBifurcationsAndChaosInEngineering, mohlerDyanmicsAndControlPartOne} helps explain the erratic functioning of stressed dynamical systems such as the power grid, and the theory of Critical Slowing Down \cite{schefferEarlyWarningSignalsForCriticalTransitions} lists tangible quantitative analysis tools which can help us detect an impending `bifurcation' (blackout) in the power grid.
\end{frame}

\begin{frame}{Introduction}
		In this thesis, we first investigate various real-world grid frequency time series archives on their robustness against minor disturbances and any kind of long-standing stability problems in them, through the use of bulk distribution probability density functions and autocorrelation decay plots. We refer to this analysis as Offline/Postmortem analysis as the input data used is sampled for a long period (several months or years). Next, we investigate the effectiveness of two statistical parameters computed in real-time listed out as per the theory of Critical Slowing Down, namely Autocorrelation and Variance as Early Warning Sign Indicators of an approaching bifurcation in a power grid. We label this analysis as the Online/Real-time analysis as the input data here is only in instantaneously available from a stream of PMU data.
\end{frame}	