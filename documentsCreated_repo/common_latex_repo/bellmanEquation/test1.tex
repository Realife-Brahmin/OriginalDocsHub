\documentclass{article}

\usepackage{tikz}
\usetikzlibrary{shapes,arrows}
\begin{document}
	
	
	\tikzstyle{block} = [draw, fill=blue!20, rectangle, 
	minimum height=3em, minimum width=6em]
	\tikzstyle{sum} = [draw, fill=blue!20, circle, node distance=1cm]
	\tikzstyle{input} = [coordinate]
	\tikzstyle{output} = [coordinate]
	\tikzstyle{pinstyle} = [pin edge={to-,thin,black}]
	
	% The block diagram code is probably more verbose than necessary
	\begin{tikzpicture}[auto, node distance=2cm,>=latex']
		% We start by placing the blocks
		\node [sum] (sum) {};
		\node [block, minimum width=2cm, minimum height=1.5cm, align=center, right of=sum, pin={[pinstyle]above:$w_k$},
		node distance=3cm] (G) {System \\ $x_{k+1} = f_k(x_k, u_k, w_k)$};
		% We draw an edge between the controller and G block to 
		% calculate the coordinate u. We need it to place the measurement block. 
		\node [output, right of=G] (output) {};
		\node [block, below of=G] (H) {$\mu_k$};
		
		% Once the nodes are placed, connecting them is easy. 
		\draw [->] (sum) -- node {$e$} (G);
		\draw [->] (G) -- node [name=y] {$y$}(output);
		\draw [->] (y) |- (H);
		\draw [->] (H) -| node[pos=0.99] {} 
		node [near end] {$y_m$} (sum);
	\end{tikzpicture}
	
\end{document}