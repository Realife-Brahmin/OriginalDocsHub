\documentclass{article}
\usepackage[utf8]{inputenc}
\usepackage{geometry}
\geometry{
    a4paper,
    total={170mm,257mm},
    left=20mm,
    top=20mm,
}
\usepackage{hyperref}

\title{Mentors Without Borders - Scientific Computing in Julia}
\author{Aryan Ritwajeet Jha}
\date{}

\begin{document}

\maketitle

\section*{FAQ for Scientific Computing in Julia}

\section{What is Scientific Computing?}
\textit{Scientific computing} refers to the use of advanced computing capabilities to understand and solve complex scientific problems. It involves mathematical modeling, numerical analysis, and the use of computers to analyze and simulate scientific data. Basically, it is like regular programming, but with a higher emphasis on performance.

\section{Why Julia for Scientific Computing?}
\textit{Julia} is a high-level, high-performance programming language designed for technical computing. It combines the ease of use of Python or R with the speed of C++, making it ideal for tasks in scientific computing. Key features include:
\begin{itemize}
    \item High performance, approaching that of statically-typed languages like C.
    \item Dynamic typing and interactive use, similar to Python or MATLAB.
    \item An extensive mathematical function library.
    \item Very easy to setup on any PC, with a single \href{https://julialang-s3.julialang.org/bin/winnt/x64/1.9/julia-1.9.4-win64.exe}{install} .
\end{itemize}

\section{What is the intended outcome from this `course'?}
\begin{itemize}
    \item Practical programming experience for my students: To encourage my students to develop analytical skills to disect a given problem at hand, and programatically implement solutions of their own to it.
    \item For my students to walk into a technical interview, or apply to a higher studies programme with confidence in their fundamental programming knowledge.
\end{itemize}

\section{What is this `course'?}
This course is based on \href{https://benlauwens.github.io/ThinkJulia.jl/latest/book.html}{Think Julia - How to Think Like a Computer Scientist} by Ben Lauwens, a great book for beginners and experienced programmers alike. This book is available online for FREE, although a hard copy may be purchased if desired (not required for the course). Students will be assigned exercise problems from this book, which are meant to expose them to various data structures, and the nuances of using them when dealing with tradeoffs between speed, scalability and implementation complexity.

\section{How will this course be delievered?}

\textbf{Contact Hours}: One hour per week via Zoom, Saturdays 12 Noon Pacific Time, which are meant to entertain doubts from the exercise problems, or any other troubleshooting. 

\textbf{Correspondence}: Slack Channels for any discussion throughout the week.

In this course, I do NOT:
\begin{itemize}
    \item Deliver lectures
\end{itemize}

In this course, I do:
\begin{itemize}
    \item Entertain doubts regarding exercise problems, or any programming problems in general
    \item Point students to other excellent resources available online for free, if they're interested in lecture based courses.
    \item Take feedback from my students regarding their current progress, and if they're having any difficulties, set up additional sessions to troubleshoot them
    \item Provide encouragement and guidance regarding careers in tech to students interested in the same
\end{itemize}

\section{Who Should Take This Course?}
Normally, studying scientific computing requires \textit{some} prior programming experience. However, I've designed the course such that even first time programmers can  
This course is designed for:
\begin{itemize}
    \item If you wanna break into tech.
    \item If you're already in tech but are uncertain about any of these terms: data science, data analysis, artificial intelligence, machine learning, data structure and algorithms, etc. : the stuff covered in this course will enable you to branch out in any of these directions
    \item You wish to pursue a higher degree in STEM
\end{itemize}

\section{Who may Avoid This Course?}
The course might not be the best fit for you if:
\begin{itemize}
    \item You want to do only frontend development work such as web development, app development, etc. Making webpages or applications is not an intended outcome of this course. 
\end{itemize}

\section{Are There Prerequisites for This Course?}
\textbf{Knowledge}: High School Maths including Algebra, Word Problems, knowledge of Decimals and Fractions, etc. \\
\textbf{Infrastructure}: A Computer with Internet Access.

\end{document}
