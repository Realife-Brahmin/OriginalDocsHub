\section{A Comparison of select Correlation Coefficients}

\subsection*{Pearson's Correlation Coefficient}

Pearson's Correlation Coefficient, often denoted using the parameter $\rho$, is one of the go-to methods for computing the degree of correspondence between two datasets or time-series, etc. The value of $\rho$ varies between positive and negative unity, i.e. $-1 \leq \rho \leq +1$, where the more the value is skewed towards either extreme end, the more correlated the two datasets or time-series are. The sign determines if the correlation is positive or negative. A value of $\rho$ closer to $0$ implies little to no correlation between the two variables.

\Cref{tab:pearsonCorrCoeff01} gives a quick overview of this correlation coefficient.

\begin{table}[!ht]
	\centering
	\caption{Interpretation of different values of the Pearson's $\rho$ Correlation Coefficient.}
	\label{tab:pearsonCorrCoeff01}
	
	\begin{tabular}{ccc}
		\toprule
		\begin{tabular}{l}
			Pearson's Correlation\\
			Coefficient $\rho$
		\end{tabular} & Interpretation & \begin{tabular}{l}
			Examples of two variables \\
			specific to Power Systems \\
			following the relationship
		\end{tabular}\\ 
		\midrule
		$\rho \in (0, 1]$ & 
		\begin{tabular}{l}
			A positive value of \\
			the correlation coefficient\\
			implies that both the\\
			variables change in the\\
			same `direction'.
		\end{tabular} & 
		\begin{tabular}{l}
				\begin{tabular}{l}
					Population of a city \\
					\begin{tabular}{c}
						vs
					\end{tabular}\\
					Daily Power Demand
				\end{tabular}\\
				\midrule
				\begin{tabular}{l}
					Transmission Line Length\\
					\begin{tabular}{c}
						vs
					\end{tabular}\\
					Line Losses
				\end{tabular}
		\end{tabular}\\
		\midrule
		 $\rho \approx 0$ & 
		 \begin{tabular}{l}
		 	A near zero value of \\
		 	the correlation coefficient\\
		 	implies that both the\\
		 	variables are\\
		 	uncorrelated. In\\
		 	some special cases,\\
		 	such as the case of\\
		 	jointly Gaussian\\
		 	variables, a zero\\
		 	correlation implies the\\
		 	independence of the two\\
		 	variables. \cite{zeroCorrelationImpliesIndependenceForGaussianDistribution}
		 \end{tabular} & 
		 \begin{tabular}{l}
		 	\begin{tabular}{l}
		 		Nominal Frequency of Grid\\
		 		\begin{tabular}{c}
		 			vs
		 		\end{tabular}\\
		 		Wattage of end-user electronics
		 	\end{tabular}\\
		 	\midrule
		 	\begin{tabular}{l}
		 		Nominal Frequency \\
		 		\begin{tabular}{c}
		 			vs
		 		\end{tabular}\\
		 		Setting of a Distance Relay
		 	\end{tabular}
		 \end{tabular}\\
	 \midrule
	 $\rho \in (-1, 0)$ & 
	 \begin{tabular}{l}
	 	A negative value of \\
	 	the correlation coefficient\\
	 	implies that the\\
	 	variables change in\\
	 	opposite `directions'.
	 \end{tabular} & 
	 \begin{tabular}{l}
	 	\begin{tabular}{l}
	 		Maximum Loadability of a Line\\
	 		\begin{tabular}{c}
	 			vs
	 		\end{tabular}\\
	 		Line Reactance
	 	\end{tabular}\\
	 	\midrule
	 	\begin{tabular}{l}
	 		Bus Voltage\\
	 		\begin{tabular}{c}
	 			vs
	 		\end{tabular}\\
 			Line current flowing \\
 			into that Bus
	 	\end{tabular}
	 \end{tabular}\\
 	\bottomrule
	\end{tabular}
\end{table}
\subsection*{Spearman's Ranked Correlation Coefficient}

\subsection*{Kendall's Correlation Coefficient}

\subsection*{Modified Kendall's Correlation Coefficient}