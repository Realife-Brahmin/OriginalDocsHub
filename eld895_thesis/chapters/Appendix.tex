\appendix

\section{Effect of sampling duration on consistency of offline analysis of data}
\label{app:effectOfSamplingDuration}

Different grids can have their own set of `events', such as firing up of boilers or other kinds of switching events, or `cycles' of changes, such as sub-hourly power dispatches, semi-diurnal variation in solar generation, hourly wind power fluctuations, etc. A question then arises: Can the `statistical nature' of a grid be really generalized if it itself doesn't show a constant characteristic in its dynamics? The answer is: Yes, but only if a `sufficient' duration of data has been collected for analysis, such that all kinds of cyclical variations are `averaged-out' over the duration, displaying a somewhat consistent statistical signature irrespective of the actual time the analysis was made or data collected from.

Data for different years or months of select grids (Great Britain, France RTE, Nordic, Japan) mentioned in Table \ref{tab:realGridSamplingData} was compared in terms of their Fixed Time Autocorrelation plots (Autocorrelation Decay Curves) and it was found that barring some small vertical shifts among the autocorrelation values, the trends were consistently displayed for different months or years. 

Data for five different days for the Indian grid (NRLDC) was also compared in a similar fashion. The plots were inconsistent and therefore a day's worth of data could be considered insufficient to average-out all the dynamic differences in a grid's statistical signature.

Thus a minimum of one month could be considered a sufficient duration to model the bulk characteristics of a grid.




