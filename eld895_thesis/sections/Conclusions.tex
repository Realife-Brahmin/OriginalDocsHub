\section[Conclusions]{Conclusions}
\label{sec:concl}

\noindent\textbf{Offline/Postmortem Analysis}
\begin{itemize}
	\item Frequency time-series for months/years of data obtained from various real-world grids were converted into probability distribution function plots and autocorrelation decay plots ($c(\tau)$ vs $\tau$ plots).
	\item Visual inspection of the probability distribution function plots provided many insights into the presence of long-standing steady-state instabilities in the grid as well as the grid's resilience against any additional instability causing agents. Generally the more robust grids (such as the RTE (France) and Continental European grids) are mostly Gaussian except that they have heavier tails, whereas the smaller or island grids (such as the Mallorcan (Spain) grid) can have multiple peaks, skewed distributions and thus an overall visible deviation from Gaussianity which explains  
\end{itemize}